\section{Dieny}

Dieny jest to grupa organicznych związków chemicznych, węglowodory nienasycone, w których występują dwa wiązania podwójne między atomami węgla.

W zależności od liczby wiązań pojedynczych znajdujących się pomiędzy wiązaniami podwójnymi w łańcuchu węglowym rozróżnia się trzy rodzaje dienów:

\subsection{Skumulowane}

Zwane są one allenami, w których wiązania podwójne sąsiadują ze sobą.
\vspace{0.5cm}

\chemfig{-[7]=[1]=[7]-[1]-[7]}

\subsection{Sprzężone}

Wiązania podwójne są w nich rozdzielone wiązaniem pojedynczym.
\vspace{0.5cm}

\chemfig{-[7]=[1]-[7]=[1]-[7]}

\subsection{Izolowane}

Pomiędzy wiązaniami podwójnymi występują conajmniej dwa wiązania pojedyncze.
\vspace{0.5cm}

\chemfig{-[7]=[1]-[7]-[1]=[7]}