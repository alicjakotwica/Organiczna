\section{Elektrofile, nukleofile i wolne rodniki}

\subsection{Elektrofil}

Elektrofil, inaczej czynnik elektrofilowy jest to cząsteczka lub grupa, w której występuje niedomiar elektronów i w odpowiednich warunkach jest w stanie je przyjąć, czyli być ich akceptorem.

Elektrofilami są wszystkie kwasy, zarówno zgodne z definicją Brønsteda, jak i Lewisa. Oprócz tego mogą to być także cząsteczki, które nie wykazują żadnych właściwości kwasowych, lecz tylko mają "zwykły" deficyt elektronów – pojęcie elektrofila jest więc szersze od pojęcia kwasu.

\subsection{Nukleofil}

Nukleofil to cząsteczka lub grupa, w której występuje nadmiar elektronów i w odpowiednich warunkach może być ich donorem.

W ogólnym rozumieniu tego terminu nukleofilami są wszystkie zasady Lewisa, a zatem i Brönsteda. Należy jednak zwrócić uwagę, że w chemii organicznej słowo nukleofil ma zwykle węższe znaczenie i odnosi się do cząsteczek atakujących atomy elektrofilowe, a niekoniecznie silnie wiążących protony.

\subsection{Wolny rodnik}

Rodniki (dawniej wolne rodniki) są to atomy lub cząsteczki zawierające niesparowane elektrony.

Typowy przykład reakcji, w wyniku której powstają rodniki to np. rozpad cząsteczki chloru $Cl_2$ pod wpływem działania światła ultrafioletowego:

\schemestart
    \chemfig{Cl_2}
    \+
    \chemfig{hv}
    \arrow{->}
    \chemfig{2Cl^\cdot}
\schemestop