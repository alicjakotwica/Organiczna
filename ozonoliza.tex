\section{Ozonoliza}

Ozonolizę prowadzi się traktując badany alken ozonem $(O_3)$ a następnie redukuje się powstały produkt pośredni za pomocą cynku w środowisku jonu hydroniowego. Metoda ta znajduje zastosowanie w określaniu położenia wiązania podwójnego w cząsteczce.
\vspace{1cm}

\schemestart
    \chemfig{-[1]=[7]-[1]-[7]}
    \arrow{->[$O_3$][$Zn$,$H_3O^+$]}
    \chemfig{-[1]=[7]O}
    \+
    \chemfig{O=[7]-[1]-[7]}
\schemestop