\section{Reakcje węglowodorów}

Wśród reakcji, którym ulegają węglowodory można wyróżnić cztery podstawowe typy: 

\begin{enumerate}
    \item substytucja- ulegają jej tylko węglowodory nasycone (alkany) i aromatyczne. Reakcja ta zachodzi pod wpływem naświetlania. Przykładem substytucji może być reakcja podstawienia jednego atomu wodoru przez atom chloru w cząsteczce metanu. W reakcji tej otrzymujemy chlorowcopochodną metanu i kwas solny. Reakcję taką przedstawia równanie:

    \schemestart
        \chemfig{CH_3}
        \+
        \chemfig{Cl_2}
        \arrow{->}
        \chemfig[atom sep=2em]{CH_3-Cl}
        \+
        \chemfig{HCl}
    \schemestop
    \item addycja- ulegają jej wyłącznie węglowodory aromatyczne i nienasycone. Addycja polega na rozerwaniu wiązania podwójnego i przyłączeniu się cząsteczki, która bierze udział w tej reakcji. Nie powstają żadne produkty uboczne. Do węglowodorów mogą się przyłączać:

    \begin{itemize}
        \item cząsteczki homoatomowe (zawierające dwa takie same atomy, np. Cl2, Br2 i H2)
        
        \schemestart
            \chemfig[atom sep=2em]{CH_2=CH-CH_3}
            \+
            \chemfig{H_2}
            \arrow{->}
            \chemfig[atom sep=2em]{CH_3-CH_2-CH_3}
        \schemestop

        Wiązanie podwójne między pierwszymi dwoma atomami węgla w cząsteczce propenu uległo rozerwaniu i przyłączyły się dwa atomy wodoru. Powstał propan.
        \item cząsteczki heteroatomowe (np. HCl, HI, HBr)
        
        W tym przypadku sytuacja jest bardzo podobna. Wiązanie podwójne ulega rozerwaniu, a atom wodoru przyłącza się do tego atomu węgla w cząsteczce węglowodoru, przy którym przed reakcją było więcej atomów wodoru (reguła Markownikowa). Gdy wodór się tak przyłączy powstanie wówczas produkt główny. Należy jednak pamiętać, że w reakcji tej powstanie także produkt uboczny, który będzie wynikiem podstawienie wodoru do atomu węgla, który przed reakcją zawierał mniejszą ilość atomów wodoru. Obydwa produkty powstaną równocześnie, jednak produktu głównego jest znacznie więcej.

        \schemestart
            \chemfig[atom sep=2em]{CH_3CH=CH_2}
            \+
            \chemfig{HBr}
            \arrow{->}
            \chemfig[atom sep=2em]{CH_3CHBr-CH_3}
        \schemestop
    \end{itemize}
    \newpage
    \item eliminacja- reakcja, w czasie której następuje oderwanie dwóch podstawników przy sąsiednich atomach węgla. W wyniku eliminacji (oderwania) dwóch atomów wodoru od sąsiednich atomów węgla w alkenie, powstanie alkin- pomiędzy atomami węgla powstaje wiązanie potrójne.

    \schemestart
        \chemfig[atom sep=2em]{C(-[3]H)(-[5]H)=C(-[1]H)(-[7]H)}
        \arrow{->}
        \chemfig[atom sep=2em]{H-C~C-H}
        \+
        \chemfig[atom sep=2em]{H_2}
    \schemestop
    \item polimeryzacja- ulegają jej węglowodory nienasycone. Reakcja ta polega na łączeniu się pojedynczych cząsteczek (monomerów) w produkty o bardzo dużej masie (polimery). Reakcja ta przebiega w obecności katalizatora.
\end{enumerate}